\documentclass{beamer}

\usepackage[spanish]{babel}
\usepackage[utf8x]{inputenc}
\usepackage{tabularx}
\usepackage{graphicx}
\usepackage{url}
\usepackage{hyperref}
\hypersetup{colorlinks,
  citecolor=blue,
  linkcolor=blue,
  urlcolor=blue}

\usetheme{Warsaw}
\usecolortheme{default}

\begin{document}

\title[Electiva Base de Datos I]{ELECTIVA BASE DE DATOS I.}
\subtitle[Presentación]{Presentación}
\author[M.Sc. Andrés Calderón]{Andrés O. Calderón Romero}
\institute[UdeNar]{Programa de Ingeniería de Sistemas, Universidad de Nariño}
\date[\today]{\today}

\begin{frame}
  \titlepage
\end{frame}

\AtBeginSection[]
{
\begin{frame}{Contenido}
\tableofcontents[currentsection, hideallsubsections]
\end{frame}
}

\section{Introducción}
\begin{frame}\frametitle{Introducción}
  \begin{block}{Temática}
    Descubrimiento de conocimiento en bases de datos.
  \end{block}
  \begin{block}{Enfoque}
    Bodegas de datos.
  \end{block}
\end{frame}

\begin{frame}\frametitle{Introducción}
  \begin{block}{Objetivo}
    Introducirnos en el uso y aplicación de herramientas prácticas para el diseño de bodegas de datos haciendo énfasis en soluciones de código abierto.
  \end{block}
\end{frame}

\section{Contenido}
\begin{frame}\frametitle{Contenido}
  \begin{itemize}
    \item Introducción a Pentaho.
      \begin{itemize}
	\item Conceptos generales.
	\item Instalación y configuración del servidor.
	\item El proceso de inteligencia de negocios en Pentaho.
	\item Algunos ejemplos.
      \end{itemize}
    \item Modelamiento dimensional y diseño de bodegas de datos.
      \begin{itemize}
	\item Introducción a bodegas de datos.
	\item Modelamiento del negocio usando esquemas de estrella.
	\item Proceso de diseño de almacenes de datos.
	\item Caso de estudio: World Class Movies.
      \end{itemize}
  \end{itemize}
\end{frame}

\begin{frame}\frametitle{Contenido}
  \begin{itemize}
    \item Integración de datos y procesos ETL.
      \begin{itemize}
	\item Introducción a Pentaho Data Integration (PDI).
	\item Diseñando soluciones PDI.
	\item Desarrollando soluciones PDI.
      \end{itemize}
    \item Aplicaciones de inteligencia de negocios.
      \begin{itemize}
	\item Usando las herramientas de generación de reportes.
	\item Soluciones OLAP usando Pentaho Analysis Services.
	\item Minería de datos con WEKA.
	\item Construyendo paneles de mando (Dashboards).
      \end{itemize}
  \end{itemize}
\end{frame}

\section{Recursos}
\begin{frame}\frametitle{Recursos}
  \begin{block}{Bibliografía, web y listas de correo}
    \begin{itemize}
      \item Bibliografía
	\begin{itemize}
	  \item Pentaho Solutions: Business Intelligence and Data Warehousing with Pentaho and MySQL (Bouman and Dongen, 2009).
	  \item The Data Warehouse Toolkit Second Edition (Kimball and Ross,2002).
	  \item Pentaho Kettle Solutions: Building Open Source ETL Solutions with Pentaho Data Integration (Casters et al, 2010).
	  \item The Data Warehouse ETL Toolkit (Kimball and Caserta, 2004).
	  \item Pentaho 3.2 Data Integration: Beginner\'s Guide (Roldán, 2010).
	  \item Pentaho Reporting 3.5 for Java Developers (Gorman, 2009).
	\end{itemize}
    \end{itemize}
  \end{block}
\end{frame}

\begin{frame}\frametitle{Recursos}
  \begin{block}{Bibliografía y recursos web}
    \begin{itemize}
      \item Recursos web
	\begin{itemize}
	  \item \href{http://community.pentaho.com/}{\url{http://community.pentaho.com/}}
	  \item \href{http://en.wikipedia.org/wiki/Business\_intelligence}{\url{http://en.wikipedia.org/wiki/Business\_intelligence}}
	  \item \href{http://es.wikipedia.org/wiki/}{\url{http://es.wikipedia.org/wiki/Almacen\_de\_datos}}
	  \item \href{http://www.kimballgroup.com}{\url{http://www.kimballgroup.com}}
	  \item \href{http://b-eye-network.com}{\url{http://b-eye-network.com}}
	  \item \href{http://www.tdwi.org}{\url{http://www.tdwi.org}}
	\end{itemize}
    \end{itemize}
  \end{block}
\end{frame}

\begin{frame}\frametitle{Recursos}
  \begin{block}{Bibliografía y recursos web}
    \begin{itemize}
      \item Foros y wikis (Disponibles en \href{http://community.pentaho.com/}{\url{http://community.pentaho.com/}})
	\begin{itemize}
	  \item \href{http://forums.pentaho.com/forumdisplay.php?73-BI-Platform}{BI Platform}
	  \item \href{http://forums.pentaho.com/forumdisplay.php?135-Pentaho-Data-Integration-Kettle}{PDI}
	\end{itemize}
    \end{itemize}
  \end{block}
\end{frame}

\section{Evaluación}
\begin{frame}\frametitle{Evaluación}
  \begin{block}{Mecanismos de evaluación}
    \begin{itemize}
      \item Exposiciones (30\%)
      \item Talleres (30\%)
      \item Proyecto Final (40\%)
    \end{itemize}
  \end{block}
\end{frame}

\section{Contacto}
\begin{frame}\frametitle{Contacto}
  \begin{block}{Atención a estudiantes}
    \centering
    \vspace{5mm}
    \href{mailto://bodegasdedatos.udenar@gmail.com}{\url{bodegasdedatos.udenar@gmail.com}}\\
    Grupo de Investigación Aplicada a Sistemas - GRiAS\\
    Laboratorios de Ingeniería - Universidad de Nariño
  \end{block}
\end{frame}

% \begin{frame}\frametitle{}
% 
% \end{frame}

\end{document}
