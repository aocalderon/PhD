\documentclass[10pt]{scrartcl}
\usepackage[hmargin=2.5cm,vmargin=3cm]{geometry}
\usepackage[utf8]{inputenc}
\usepackage{graphicx}
\usepackage{hyperref}
\hypersetup{
colorlinks=false,
hidelinks
}
\usepackage{minted}
\usepackage{xcolor}
\usepackage{amsmath}

%opening
\title{Max Min Average Temperature by States - MapReduce Project/CS236}
\author{Andres Calderon - SID:861243796}

\begin{document}

\maketitle
\section{Max Min Average Temperature by States}

\subsection{Username and node number}
My username is acald013 and I used the z7 node.

\subsection{Overall description}
My main idea was to combine the required data in one file where station id is the key. Then a reduce job put the station id together and we can emulate an inner join.  After that, different map and reduce jobs organize the data and compute aggregates (average, max and min).  Finally, a mapper will format and sort the final result.

\subsection{Description of mapreduce jobs}
\subsubsection{First job}\label{sec:first-job}
The first job will read the files for the stations and recordings and apply StationMapper and DataMapper respectively. The mappers extract just the required data if they pass some conditions.  For example, in the case of the recordings it just reads records where country is `US' and state is not empty.  The output of the mappers will be a $<key, value>$ pair where the key is the station id and the value could be:
\begin{enumerate}
 \item a month and temperature record for that station (it comes from the recording files and is marked with a `D'), or
 \item the state where the station is located (it comes from the location file and is marked with a `S').
\end{enumerate}

An example of the output of the mappers can be seen in figure \ref{fig:map1}.  Then, JoinReducer is called in this job to read the mappers' output and combine the data by station id. For each station we will have a set of samples (month and temperature) and the state where the sample belongs to. The output of the job will be a new $<key, value>$ file where the key will be the combination of state and month and the value will be its temperature. Figure \ref{fig:reduce1} illustrates a possible output.  

\begin{figure}
\centering
  \inputminted[
    fontsize=\footnotesize,
    tabsize=2,
    breaklines,
    framesep=10pt,
    frame=single
    ]{bash}{map1.txt}
\caption{Output of \texttt{Data} and \texttt{Station} \texttt{Mappers}.}\label{fig:map1}
\end{figure}

\begin{figure}
\centering
  \inputminted[
    fontsize=\footnotesize,
    tabsize=2,
    breaklines,
    %linenos,
    framesep=10pt,
    frame=single
    ]{bash}{reduce1.txt}
\caption{Output of \texttt{JoinReducer}.}\label{fig:reduce1}
\end{figure}

\subsubsection{Second job}
The second job use a simple mapper (\texttt{FileMapper}) to read the last output and \texttt{AverageReducer} to compute the average aggregation.  It take advantage that the reducer collects all the values with same State-Month combination and compute the average for those values.  The reducer will map the output using the state as key and a concatenation of month and average temperature as value.  Figure \ref{fig:reduce2} shows an example of the partial result.

\begin{figure}
\centering
  \inputminted[
    fontsize=\footnotesize,
    tabsize=2,
    breaklines,
    %linenos,
    framesep=10pt,
    frame=single
    ]{bash}{reduce2.txt}
\caption{Output of \texttt{AverageReducer}.}\label{fig:reduce2}
\end{figure}

\subsubsection{Third job}
The third use again \texttt{FileMapper} to read the last output. The reduce job (\texttt{MaxMinReducer}) will collect the month and its average for each state. Then, it will select the maximum and minimum value and compute the difference.  For each case, it will extract the associated months and put them in the output.  The job will map the output using the state as key and a concatenation of the maximum temperature, the month for the maximum temperature, the minimum temperature, the month for the minimum temperature and the difference as value.  Figure \ref{fig:reduce3} shows an partial example.

\begin{figure}
\centering
  \inputminted[
    fontsize=\footnotesize,
    tabsize=2,
    breaklines,
    %linenos,
    framesep=10pt,
    frame=single
    ]{bash}{reduce3.txt}
\caption{Output of \texttt{MaxMinReducer}.}\label{fig:reduce3}
\end{figure}

\subsubsection{Fourth job}\label{sec:fourth-job}
The final job read the last output using \texttt{SortMapper}. This map uses a custom implementation (\texttt{SortableKey}) of the \texttt{WritableComparable} interface.  This implementation allow to map an output by State and Difference (of temperature).  This class implements the methods \texttt{toString()}, to print just the State in the key, and \texttt{compareTo()}, to force the reducer to sort the key by the difference.  As the intention of the job is just to sort the records it does not call a particular reducer, so the default reducer will pass the same records from the mappers but in ascending order.  Figure \ref{fig:reduce4} shows the final result. 

\begin{figure}
\centering
  \inputminted[
    fontsize=\footnotesize,
    tabsize=2,
    breaklines,
    %linenos,
    framesep=10pt,
    frame=single
    ]{bash}{reduce4.txt}
\caption{Output of \texttt{SortMapper}.}\label{fig:reduce4}
\end{figure}

\vspace{\baselineskip}
\noindent The four jobs take around 1m08s to complete\footnote{See details at \url{http://www.cs.ucr.edu/~acald013/MP_Output.txt}.}.

\subsection{How to approach the join}
Section \ref{sec:first-job} and figures \ref{fig:map1} and \ref{fig:reduce1} explain my approximation to deal with the requested join.

\subsection{Possible extra-credit}
I would like to put into consideration the implementation of a custom WritableComparable explained in section \ref{sec:fourth-job} as extra-credit.

\end{document}
