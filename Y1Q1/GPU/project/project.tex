\documentclass[10pt]{scrartcl}
\usepackage[hmargin=2.5cm,vmargin=3cm]{geometry}
\usepackage[utf8]{inputenc}
\usepackage[square]{natbib}
\usepackage{graphicx}
\usepackage{hyperref}
\hypersetup{
colorlinks=false,
hidelinks
}
\usepackage{minted}
\usepackage{xcolor}
\usepackage{amsmath}

%opening
\title{Final Project}
\author{Andres Calderon - SID:861243796}

\begin{document}

\maketitle
%\thispagestyle{empty}

\section{Motivation}
\begin{enumerate}
 \item What is the problem you would like to solve?
 
Recently increase use of location-aware devices (such as GPS, Smart-phones and RFID tags) has allowed the collection of a vast amount of data with a spatial component linked to them.  Different studies have focused in analyzing and mining this kind of collections \citep{leung_knowledge_2010}\citep{Miller_2001}.  In this area, trajectory datasets have emerged as an interesting field where diverse kind of patterns can be identified \citep{zheng_computing_2011}\citep{vieira_spatio-temporal_2013}.  For instance, authors have proposed techniques to discover motion spatial patterns such as moving clusters\citep{kalnis_discovering_2005}, convoys\citep{jeung_discovery_2008} and flocks \citep{benkert_reporting_2006}\citep{gudmundsson_computing_2006}.  In particular, \citep{vieira_-line_2009} and \citep{turdukulov_visual_2014} propose two novel algorithms to find moving flock patterns in very large spatio-temporal datasets.  
 
 \item Why is it important?
 
A flock pattern is defined as a group of entities which move together for a defined lapse of time \citep{benkert_reporting_2006}.  Applications to this kind of patterns are diverse and range from surveillance to integrated transport systems.  For example, \citep{turdukulov_visual_2014} explore the finding of this class of patterns to discover similarities between tropical cyclone paths. Also, \citep{calderon_mining_2011} finds moving flock patterns in iceberg trajectories to understand their movement behavior and how they related to changes in ocean's currents. 
 
 \item What is your plan/outline of your solution?
 
The algorithms proposed by \citep{vieira_-line_2009} and \citep{turdukulov_visual_2014} share the same initial strategy to detect flock patterns.  In that, first they find clusters of points which could be close enough to initiate a flock for each time interval.  This is a costly operation due to the large number of points and intervals to be analyzed.  The technique uses a grid-based index and a stencil (see figure~\ref{fig:grid}) to speed up the process but the complexity is still high.  My plan is to allow  individual threads to compute each of the stencils in the grid in parallel.

\begin{figure}[h!]
 \centering
 \includegraphics[width=0.25\textwidth]{figures/bfe.png}
 \caption{Grid-based index used in \citep{vieira_-line_2009}.}
 \label{fig:grid}
\end{figure}


\end{enumerate}

\section{Data}
%  \begin{figure}[h!]
%   \centering
%   \includegraphics[width=0.7\textwidth]{./grid}
%   \caption{Data points.}\label{fig:grid}
%  \end{figure}
\includegraphics[width=\textwidth]{./grid}

\section{Code}
Full code and other materials are available at \cite{github}.

\subsection{bfe.cu}
  \inputminted[
  fontsize=\footnotesize,
  tabsize=2,
  breaklines,
  linenos
  ]{c}{bfe.cu}

\subsubsection{kernel.cu}
  \inputminted[
  fontsize=\footnotesize,
  tabsize=2,
  breaklines,
  linenos
  ]{c}{kernel.cu}

\section{Output}
  \inputminted[
  fontsize=\footnotesize,
  tabsize=2,
  breaklines,
  linenos
  ]{bash}{outputs.sh}

\section{Profiler}
 \begin{figure}[h!]
  \centering
  \includegraphics[width=\textwidth]{figures/nvvp_T1-E2K-M3.png}
  \caption{NVVP performance analysis for T1-E2K-M100.}\label{fig:nvvp_T1-E2K-M100}
 \end{figure}

 \begin{figure}[h!]
  \centering
  \includegraphics[width=\textwidth]{figures/nvvp_T120-E2K-M100.png}
  \caption{NVVP performance analysis for T120-E2K-M100.}\label{fig:nvvp_T120-E2K-M100}
 \end{figure}

 \begin{figure}[h!]
  \centering
  \includegraphics[width=\textwidth]{figures/nvvp_T120-E1K-M100.png}
  \caption{NVVP performance analysis for T120-E1K-M100.}\label{fig:nvvp_T120-E1K-M100}
 \end{figure}

 \begin{figure}[h!]
  \centering
  \includegraphics[width=\textwidth]{figures/nvvp_T120-E800-M100.png}
  \caption{NVVP performance analysis for T120-E800-M100.}\label{fig:nvvp_T120-E800-M100}
 \end{figure}

 \begin{figure}[h!]
  \centering
  \includegraphics[width=\textwidth]{figures/nvvp_T120-E600-M100.png}
  \caption{NVVP performance analysis for T120-E600-M100.}\label{fig:nvvp_T120-E600-M100}
 \end{figure}

\bibliographystyle{plainnat}
\bibliography{gpu}

% \begin{thebibliography}{9}
%   \bibitem{github} 
%   Andres Calderon.
%   \textit{GitHub Personal Repository}, 2015. 
%   \url{https://github.com/aocalderon/PhD/tree/master/Y1Q1/GPU/lab3}.
% \end{thebibliography}

\end{document}
