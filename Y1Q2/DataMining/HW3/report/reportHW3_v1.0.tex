\documentclass{article}

\usepackage{alltt}
\usepackage{graphicx}
\usepackage{minted}
\usepackage{xcolor}
\usepackage{placeins}
\usepackage[margin=3cm]{geometry}
\usepackage{booktabs}
\usepackage{subcaption}
\usepackage{amsmath}
\usepackage{hyperref}
\hypersetup{
colorlinks=false,
hidelinks
}
\setlength{\parindent}{2em}
\setlength{\parskip}{0.5em}

\title{Homework 3 Report}
\author{
   Christina Pavlopoulou\\
  \small \texttt{cpavl001@ucr.edu}
  \and
   Andres Calderon\\
  \small \texttt{acald013@ucr.edu}
}
\begin{document}
\maketitle

\section{Data}
EarthExplorer\footnote{\url{http://earthexplorer.usgs.gov/}} is a website where USGS makes available a large repository of satellite imagery for scientific purposes.  We downloaded a scene from a suburb of Riverside using the High Resolution Orthoimagery dataset.  The initial scene was large, so we sample two images (figures \ref{fig:image1} and \ref{fig:image2}) which collect enough and diverse types of objects.  For this assignment, we decided to classify pools.

\begin{figure}
 \centering
 \includegraphics[trim=10 60 10 50 , clip, width=0.9\textwidth]{../figures/image1.pdf}
 \caption{First image used for training.}
 \label{fig:image1}
\end{figure}

\begin{figure}
 \centering
 \includegraphics[trim=10 60 10 50 , clip, width=0.9\textwidth]{../figures/image2.pdf}
 \caption{Second image used for validation.}
 \label{fig:image2}
\end{figure}

For the first image we collected a total of 100 points, 50 of them where pools and the remaining were not pools as our training set.

\section{Classification}
For the classification part, we trained a kNN classifier using the training set.  We used the rminer\footnote{\url{https://cran.r-project.org/web/packages/rminer/index.html}} package provided by the R project (\url{https://www.r-project.org/}) using the default parameters. 

\section{Validation}
For validation, we used the second image to collect a group of 50 points as our testing set. We created a grid of 50x50 pixels in which we checked how many of testing points belong to the same cell (figure \ref{fig:locations}). If more than one point belongs to the same cell, we keep just one of them. As a result, we kept 38 valid points. 

\begin{figure}
 \centering
 \includegraphics[trim=80 60 70 50 , clip, width=1\textwidth]{../figures/grid3.pdf}
 \caption{Locations for validation set in second image.}
 \label{fig:locations}
\end{figure}

\begin{figure}
 \centering
 \includegraphics[trim=80 150 70 50 , clip, width=0.7\textwidth]{../figures/roc.pdf}
 \caption{Locations for validation set in second image.}
 \label{fig:roc}
\end{figure}
ROC
AUC
\end{document}
