\documentclass[a4paper,10pt]{scrartcl}
\usepackage[hmargin=2.5cm,vmargin=3cm]{geometry}
\usepackage[utf8]{inputenc}
\usepackage{graphicx}
\usepackage{hyperref}
\hypersetup{
colorlinks=false,
hidelinks
}

\setlength{\parindent}{2em}
\setlength{\parskip}{0.5em}

%opening
\title{Paper Review 5/23}
\author{Andres Calderon - SID:861243796}

\begin{document}
\maketitle
\thispagestyle{empty}

\section*{On Power-Law Relationships of the Internet Topology (Faloutsos$^3$, 1999)}
This work describes the discovering of three power-laws of the Internet topology during 1998.  The authors analyze graphs derived from data which represent Internet at domain and router levels.  They propose novel graph metrics and they are able to infer relationships for domain connectivity (the rank exponent), distribution of outdegree (the outdegree exponent) and the topological properties of graphs (the eigen exponent).  In addition they propose the hop-plot approximation to quantify the connectivity and distance between the Internet nodes.

Overall, it is an enjoyable paper easy to read.  In particular, section 5 is interesting.  They discuss practical uses of the power-laws in contexts such as protocol performance, graph generation and selection, and even for growth prediction and extrapolations.  I found the explanation about the growth of Internet quite intriguing in the way that small changes can propagate through the network and have a particular impact on it.  Given that the study is out-to-date,  I would like to know if the power-laws found at the time of this work are still relevant or if, as it is mentioned in the paper, external factors have influence the growth and expansion of the Internet today in a different way.  
% \begin{thebibliography}{9}
% \bibitem{github} 
% Andres Calderon.
% \textit{GitHub Personal Repository}, 2015. 
% \url{https://github.com/aocalderon/PhD/tree/master/Y1Q1/GPU/lab3}.
% \end{thebibliography}

\end{document}
