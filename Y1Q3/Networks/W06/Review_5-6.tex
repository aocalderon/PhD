\documentclass[a4paper,10pt]{scrartcl}
\usepackage[hmargin=2.5cm,vmargin=3cm]{geometry}
\usepackage[utf8]{inputenc}
\usepackage{graphicx}
\usepackage{hyperref}
\hypersetup{
colorlinks=false,
hidelinks
}

\setlength{\parindent}{2em}
\setlength{\parskip}{0.5em}

%opening
\title{Paper Review 5/6}
\author{Andres Calderon - SID:861243796}

\begin{document}
\maketitle
\thispagestyle{empty}

\section*{Performance Analysis of the IEEE 802.11 Distributed Coordination Function (Bianchi, 1999)}
The paper presents an analytical model to describe the saturation throughput performance of DCF on the 802.11 protocol.  The paper takes into account both basic access and RTS/CTS mechanisms as techniques for packet transmission.  The performance evaluation focuses on the assumption of ideal channel conditions and finite number of terminals.

The importance of saturation throughput is that it defines the limit reached by the system throughput as the offered load increases. The analysis first uses a Markov model to find the stationary probability ($\tau$) and then express the throughput of both Basic and RTS/CTS access methods as function of $\tau$.  The paper shows strong mathematical foundation for the model and probability computation.

Overall, the paper is easy to read and provides enough information to understand the proposal.  Although the throughput analysis and model validation are described through elaborated equations, the figures and details in the evaluation are well documented and explained.  They are useful to understand the concepts and conclusions from the analysis.

% \begin{thebibliography}{9}
% \bibitem{github} 
% Andres Calderon.
% \textit{GitHub Personal Repository}, 2015. 
% \url{https://github.com/aocalderon/PhD/tree/master/Y1Q1/GPU/lab3}.
% \end{thebibliography}

\end{document}
