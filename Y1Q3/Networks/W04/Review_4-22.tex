\documentclass[a4paper,10pt]{scrartcl}
\usepackage[hmargin=2.5cm,vmargin=3cm]{geometry}
\usepackage[utf8]{inputenc}
\usepackage{graphicx}
\usepackage{hyperref}
\hypersetup{
colorlinks=false,
hidelinks
}

\setlength{\parindent}{2em}
\setlength{\parskip}{0.5em}

%opening
\title{Paper Review 4/22}
\author{Andres Calderon - SID:861243796}

\begin{document}
\maketitle
\thispagestyle{empty}

\section*{Chord: A Scalable Peer-to-peer Lookup Protocol for Internet Applications (Stoica et al., 2001)}
This paper presents Chord, a distributed lookup protocol aims to efficiently locate nodes for data storing on peer-to-peer applications.  Chord manages data and nodes using a key-value mapping approach together with consistent hashing.  As a result, Chord is highly scalable as it adapts easily when new nodes join or leave the system.  In addition, communication cost and maintenance scale appropriately according to the number of nodes.

Section III highlights the main tenets of Chord design: load balance, decentralization, scalability, adaptability and flexible naming.  Thanks to these, Chord can be used as a library and, on top of it, can be implemented a diverse kind of software applications.  Section IV explains the core of the Chord protocol: a consistent hashing improvement, scalable key location algorithms and dynamic operations and failure management.  Finally, section V presents a detailed set of simulations to evaluate the proposed protocol.

Overall, it is a paper well-organized and easy to read.  Although part of section III introduces theorems and proofs they are concise and well-supported by examples and figures.  It is also nice to see the correspondence between the concepts explained in section III and the experiments run in section IV.

% \begin{thebibliography}{9}
% \bibitem{github} 
% Andres Calderon.
% \textit{GitHub Personal Repository}, 2015. 
% \url{https://github.com/aocalderon/PhD/tree/master/Y1Q1/GPU/lab3}.
% \end{thebibliography}

\end{document}
