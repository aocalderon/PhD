%% LyX 2.1.4 created this file.  For more info, see http://www.lyx.org/.
%% Do not edit unless you really know what you are doing.
\documentclass{beamer}
\usepackage{minted}
\usepackage{animate}
\usepackage{graphicx}
\def\Put(#1,#2)#3{\leavevmode\makebox(0,0){\put(#1,#2){#3}}}
\usepackage{color}
\usepackage{tikz}

\definecolor{LightGray}{gray}{0.9}

\ifx\hypersetup\undefined
  \AtBeginDocument{%
    \hypersetup{unicode=true,
 bookmarksnumbered=false,bookmarksopen=false,
 breaklinks=false,pdfborder={0 0 0},colorlinks=false}
  }
\else
  \hypersetup{unicode=true,
 bookmarksnumbered=false,bookmarksopen=false,
 breaklinks=false,pdfborder={0 0 0},colorlinks=false}
\fi

\makeatletter
%%%%%%%%%%%%%%%%%%%%%%%%%%%%%% Textclass specific LaTeX commands.
 % this default might be overridden by plain title style
 \newcommand\makebeamertitle{\frame{\maketitle}}%
 % (ERT) argument for the TOC
 \AtBeginDocument{%
   \let\origtableofcontents=\tableofcontents
   \def\tableofcontents{\@ifnextchar[{\origtableofcontents}{\gobbletableofcontents}}
   \def\gobbletableofcontents#1{\origtableofcontents}
 }

%%%%%%%%%%%%%%%%%%%%%%%%%%%%%% User specified LaTeX commands.
\usetheme{Warsaw}
% or ...
\useoutertheme{infolines}
\addtobeamertemplate{headline}{}{\vskip2pt}

\setbeamercovered{transparent}
% or whatever (possibly just delete it)



\setbeamertemplate{footline}
{
  \leavevmode%
  \hbox{%
  \begin{beamercolorbox}[wd=.5\paperwidth,ht=2.25ex,dp=1ex,center]{title in head/foot}%
    \usebeamerfont{title in head/foot}\insertshorttitle
  \end{beamercolorbox}%
  \begin{beamercolorbox}[wd=.5\paperwidth,ht=2.25ex,dp=1ex,right]{date in head/foot}%
    \usebeamerfont{date in head/foot}\insertshortdate{}\hspace*{2em}
    \insertframenumber{} / \inserttotalframenumber\hspace*{2ex} 
  \end{beamercolorbox}}%
  \vskip0pt%
}

\makeatother

\begin{document}

\title[Simba]{Simba: Efficient In-Memory Spatial Analytics.}
\subtitle{Dong Xie, Feifei Li, Bin Yao, Gefei Li, Liang Zhou and Minyi Guo \\ \small SIGMOD'16.}
\author{Andres Calderon}
%\institute{University of California, Riverside}

\makebeamertitle

% \AtBeginSection[]{
%   \frame<beamer>{ 
%     \frametitle{Agenda}   
%     \tableofcontents[currentsubsection] 
%   }
% }

\newif\iflattersubsect

\AtBeginSection[] {
    \begin{frame}<beamer>
    \frametitle{Agenda} %
    \tableofcontents[currentsection]  
    \end{frame}
    \lattersubsectfalse
}

\AtBeginSubsection[] {
    % \iflattersubsect
    \begin{frame}<beamer>
    \frametitle{Agenda} %
    \tableofcontents[currentsubsection]  
    \end{frame}
    % \fi
    % \lattersubsecttrue
}

\begin{frame}{Agenda}
  \tableofcontents{}
\end{frame}

\section{Background}

\begin{frame}{Introduction}
  \begin{itemize}
    \item There has been an explosion in the amount of spatial data in recent years...
  \end{itemize} \vspace{3cm}
  \Put(60,0)  {\includegraphics[width=3cm, height=3cm]{Figures/autonomous_cars.jpg}}
\end{frame}

\begin{frame}[noframenumbering]{Introduction}
  \begin{itemize}
    \item There has been an explosion in the amount of spatial data in recent years...
  \end{itemize} \vspace{3cm}
  \Put(60,0)  {\includegraphics[width=3cm, height=3cm]{Figures/autonomous_cars.jpg}}
  \Put(100,40)  {\includegraphics[width=3cm, height=3cm]{Figures/survillance.png}}
\end{frame}

\begin{frame}[noframenumbering]{Introduction}
  \begin{itemize}
    \item There has been an explosion in the amount of spatial data in recent years...
  \end{itemize} \vspace{3cm}
  \Put(60,0)  {\includegraphics[width=3cm, height=3cm]{Figures/autonomous_cars.jpg}}
  \Put(100,40)  {\includegraphics[width=3cm, height=3cm]{Figures/survillance.png}}
  \Put(150,90){\includegraphics[width=3cm, height=3cm]{Figures/urban_planning.png}}
\end{frame}

\begin{frame}[noframenumbering]{Introduction}
  \begin{itemize}
    \item There has been an explosion in the amount of spatial data in recent years...
  \end{itemize} \vspace{3cm}
  \Put(60,0)  {\includegraphics[width=3cm, height=3cm]{Figures/autonomous_cars.jpg}}
  \Put(100,40)  {\includegraphics[width=3cm, height=3cm]{Figures/survillance.png}}
  \Put(150,90){\includegraphics[width=3cm, height=3cm]{Figures/urban_planning.png}}
  \Put(200,140){\includegraphics[width=3cm, height=3cm]{Figures/agricultural.png}}
\end{frame}

\begin{frame}{Introduction}
  \begin{itemize}
    \item The applications and commercial interest is clear...
  \end{itemize}
  \centering
  \includegraphics[width=0.75\linewidth]{Figures/apps.png}
\end{frame}

\begin{frame}{Introduction}
  \begin{itemize}
    \item But remember that ``Spatial is Special''...
  \end{itemize}
  \centering
  \includegraphics[clip, trim=0cm 5.75cm 0cm 0cm, width=0.75\linewidth]{Figures/logos.png}
\end{frame}

\begin{frame}[noframenumbering]{Introduction}
  \begin{itemize}
    \item But remember that ``Spatial is Special''...
  \end{itemize}
  \centering
  \includegraphics[clip, trim=0cm 2.4cm 0cm 0cm, width=0.75\linewidth]{Figures/logos.png}
\end{frame}

\begin{frame}[noframenumbering]{Introduction}
  \begin{itemize}
    \item But remember that ``Spatial is Special''...
  \end{itemize}
  \centering
  \includegraphics[clip, trim=0cm 0cm 0cm 0cm, width=0.75\linewidth]{Figures/logos.png}
\end{frame}

\begin{frame}{Introduction}
  \begin{itemize}
    \item Why do we need a new tool???
  \end{itemize}
  \centering
  \includegraphics[width=0.3\linewidth]{Figures/simba-logo.png}
\end{frame}

\begin{frame}{Introduction}
\begin{itemize}
\item Problems of Existing Systems...
  \begin{itemize}
    \item Single node database (low scalability) \\
	  \scriptsize{ArcGIS, PostGIS, Oracle Spatial.}
    \item \normalsize Disk-oriented cluster computation (low performance) \\
	  \scriptsize{Hadoop-GIS, SpatialHadoop, GeoMesa.}
    \item \normalsize No native support for spatial operators \\
	  \scriptsize{Spark SQL, MemSQL}
    \item \normalsize No sophisticated query planner and optimizer \\
	  \scriptsize{SpatialSpark, GeoSpark}
  \end{itemize}
\end{itemize}
\end{frame}

\begin{frame}{Introduction}
  \begin{itemize}
    \item Simba: \textbf{S}patial \textbf{I}n \textbf{M}emory \textbf{B}ig data \textbf{A}nalytics.
    \begin{enumerate}
     \item Extends Spark SQL to support spatial queries and offers simple APIs for both SQL and DataFrame.\pause
     \item Support two-layer spatial indexing over RDDs (low latency).\pause
     \item Designs a SQL context to run important spatial operations in parallel (high throughput).\pause
     \item Introduces spatial-aware and cost-based optimizations to select good spatial plans.
    \end{enumerate}
  \end{itemize}
\end{frame}

\begin{frame}{Introduction}
  \centering
  \includegraphics[clip, trim=1cm 20cm 10cm 1cm, page=3, width=0.9\linewidth]{simba_paper}
\end{frame}

\section{Simba Architecture Overview}

\begin{frame}[fragile]{Spark SQL Overview}
  Spark SQL is Apache Spark's module for working with structured data.\pause
  \begin{itemize}
   \item Seamlessly mixes SQL queries with Spark programs.\pause
   \item Connects to any data source the same way.\pause
   \item Includes a highly extensible cost-based optimizer (\textit{Catalyst}).\pause
   \item Spark SQL is a full-fledged query engine based on the underlying Spark core.
  \end{itemize}
\end{frame}

\begin{frame}[fragile]{Spark SQL Overview}
  \centering
  \begin{minted}[fontsize=\tiny,tabsize=8,breaklines,framesep=10pt,frame=single]{python}
	# Apply functions to results of SQL queries.
	context = HiveContext(sc)
	results = context.sql("""
				SELECT 
					* 
				FROM 
					people""")
	names = results.map(lambda p: p.name)
	# Query and join different data sources.
	context.jsonFile("s3n://...").registerTempTable("json")
	results = context.sql("""
				SELECT 
				      * 
				FROM 
				      people
				JOIN 
				      json ...""")
  \end{minted}
\end{frame}

\begin{frame}{Simba Architecture}
  Simba is an extension of Spark SQL across the system stack.
  \centering
  \begin{minipage}{.75\linewidth} 
    \begin{tikzpicture}
      \node[anchor=south west,inner sep=0] at (0,0) {\includegraphics[clip, trim=2cm 21.75cm 11.1cm 1cm, page=4, width=\linewidth]{simba_paper}};
      \draw[olive,very thick,rounded corners] (0,3.4) rectangle (9,3.85);
    \end{tikzpicture}
  \end{minipage}
  \begin{minipage}{.21\linewidth} \tiny \color{olive}1.Programming interface \\ \color{white}2.Table indexing \\ 3.Efficient spatial operators \\ 4.New query optimizations \end{minipage}  
\end{frame}

\begin{frame}[noframenumbering]{Simba Architecture}
  Simba is an extension of Spark SQL across the system stack.
  \centering
  \begin{minipage}{.75\linewidth} 
    \begin{tikzpicture}
      \node[anchor=south west,inner sep=0] at (0,0) {\includegraphics[clip, trim=2cm 21.75cm 11.1cm 1cm, page=4, width=\linewidth]{simba_paper}};
      \draw[olive,very thick,rounded corners] (0,3.4) rectangle (9,3.85);
      \draw[magenta,very thick,rounded corners] (0,2.88) rectangle (9,3.35);
    \end{tikzpicture}
  \end{minipage}
  \begin{minipage}{.21\linewidth} \tiny \color{olive}1.Programming interface \\ \color{magenta}2.Table indexing \\ \color{white}3.Efficient spatial operators \\ 4.New query optimizations \end{minipage}  
\end{frame}

\begin{frame}[noframenumbering]{Simba Architecture}
  Simba is an extension of Spark SQL across the system stack.
  \centering
  \begin{minipage}{.75\linewidth} 
    \begin{tikzpicture}
      \node[anchor=south west,inner sep=0] at (0,0) {\includegraphics[clip, trim=2cm 21.75cm 11.1cm 1cm, page=4, width=\linewidth]{simba_paper}};
      \draw[olive,very thick,rounded corners] (0,3.4) rectangle (9,3.85);
      \draw[magenta,very thick,rounded corners] (0,2.88) rectangle (9,3.35);
      \draw[red,very thick,rounded corners] (2.05,2.37) rectangle (9,2.83);
    \end{tikzpicture}
  \end{minipage}
  \begin{minipage}{.21\linewidth} \tiny \color{olive}1.Programming interface \\ \color{magenta}2.Table indexing \\ \color{red}3.Efficient spatial operators \\ \color{white}4.New query optimizations \end{minipage}  
\end{frame}

\subsection{Programming Interface}


\subsection{Indexing}


\subsection{Spatial Operations}


\subsection{Optimization}


\section{Experiments}

\begin{frame}{Setup}
\begin{itemize}
 \item Cluster of 25 nodes:
 \begin{itemize}
  \item HDD from 50GB to 200GB.
  \item RAM from 2GB to 8GB.
  \item Processors 2.2GHz to 3GHz
 \end{itemize}
 \item Single machine:
 \begin{itemize}
  \item HDD 2TB.
  \item RAM 16GB.
  \item Processor 3.4GHz.
 \end{itemize}
\end{itemize}
\end{frame}

\begin{frame}{Datasets}
\begin{itemize}
 \item Real datasets (from OpenStreetMap):
 \begin{itemize}
  \item OSM1: 164M polygons, 80GB.
  \item OSM2: 1.7B points, 52GB.
 \end{itemize}
 \item Synthetic dataset:
 \begin{itemize}
  \item SYNTH: 3.8B points, 128GB.
  \item Five different distributions.
 \end{itemize}
\end{itemize}
  \centering
  \includegraphics[width=0.5\linewidth]{Figures/img2.jpg} \\
\end{frame}

\section{Conclusions}

\begin{frame}{Conclusions}
  \begin{itemize}
    \item This paper introduced CG\_Hadoop as a scalable and efficient MapReduce library.
    \item Focused on 5 fundamental computational geometry problems...
    \begin{itemize}
      \item Polygon union, Skyline, Convex hull, Farthest and Closest Pairs.
    \end{itemize}
    \item Provided versions for Apache Hadoop and SpatialHadoop systems.
    \item Distributed approach speed up performance.
    \item Spatial partitioning allows early pruning which make it even more efficient. 
    \item Achieve up to 29x and 260x better performance.
  \end{itemize}
\end{frame}

\begin{frame}{Future ideas}
  \begin{itemize}
    \item Working on more complex operations, for example motion patterns.
    \item Explore ports to new distributed platforms such as Spark or Simba.
  \end{itemize}
\end{frame}


\subsection*{Thanks...}

\begin{frame}{}
  \centering
  \huge Thank you!!! \\
  \vspace{2cm}
  \large Do you have any question?
\end{frame}

\end{document}