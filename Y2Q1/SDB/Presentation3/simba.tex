%% LyX 2.1.4 created this file.  For more info, see http://www.lyx.org/.
%% Do not edit unless you really know what you are doing.
\documentclass{beamer}
\usepackage{minted}
\usepackage{animate}
\usepackage{graphicx}
\def\Put(#1,#2)#3{\leavevmode\makebox(0,0){\put(#1,#2){#3}}}
\usepackage{color}
\usepackage{tikz}

\definecolor{LightGray}{gray}{0.9}

\ifx\hypersetup\undefined
  \AtBeginDocument{%
    \hypersetup{unicode=true,
 bookmarksnumbered=false,bookmarksopen=false,
 breaklinks=false,pdfborder={0 0 0},colorlinks=false}
  }
\else
  \hypersetup{unicode=true,
 bookmarksnumbered=false,bookmarksopen=false,
 breaklinks=false,pdfborder={0 0 0},colorlinks=false}
\fi

\makeatletter
%%%%%%%%%%%%%%%%%%%%%%%%%%%%%% Textclass specific LaTeX commands.
 % this default might be overridden by plain title style
 \newcommand\makebeamertitle{\frame{\maketitle}}%
 % (ERT) argument for the TOC
 \AtBeginDocument{%
   \let\origtableofcontents=\tableofcontents
   \def\tableofcontents{\@ifnextchar[{\origtableofcontents}{\gobbletableofcontents}}
   \def\gobbletableofcontents#1{\origtableofcontents}
 }

%%%%%%%%%%%%%%%%%%%%%%%%%%%%%% User specified LaTeX commands.
\usetheme{Warsaw}
% or ...
\useoutertheme{infolines}
\addtobeamertemplate{headline}{}{\vskip2pt}

\setbeamercovered{transparent}
% or whatever (possibly just delete it)



\setbeamertemplate{footline}
{
  \leavevmode%
  \hbox{%
  \begin{beamercolorbox}[wd=.5\paperwidth,ht=2.25ex,dp=1ex,center]{title in head/foot}%
    \usebeamerfont{title in head/foot}\insertshorttitle
  \end{beamercolorbox}%
  \begin{beamercolorbox}[wd=.5\paperwidth,ht=2.25ex,dp=1ex,right]{date in head/foot}%
    \usebeamerfont{date in head/foot}\insertshortdate{}\hspace*{2em}
    \insertframenumber{} / \inserttotalframenumber\hspace*{2ex} 
  \end{beamercolorbox}}%
  \vskip0pt%
}

\makeatother

\begin{document}

\title[Simba]{Simba: Efficient In-Memory Spatial Analytics.}
\subtitle{Dong Xie, Feifei Li, Bin Yao, Gefei Li, Liang Zhou and Minyi Guo \\ \small SIGMOD'16.}
\author{Andres Calderon}
%\institute{University of California, Riverside}

\makebeamertitle

% \AtBeginSection[]{
%   \frame<beamer>{ 
%     \frametitle{Agenda}   
%     \tableofcontents[currentsubsection] 
%   }
% }

\newif\iflattersubsect

\AtBeginSection[] {
    \begin{frame}<beamer>
    \frametitle{Agenda} %
    \tableofcontents[currentsection]  
    \end{frame}
    \lattersubsectfalse
}

\AtBeginSubsection[] {
    % \iflattersubsect
    \begin{frame}<beamer>
    \frametitle{Agenda} %
    \tableofcontents[currentsubsection]  
    \end{frame}
    % \fi
    % \lattersubsecttrue
}

\begin{frame}{Agenda}
  \tableofcontents{}
\end{frame}

\section{Background}

\begin{frame}{Introduction}
  \begin{itemize}
    \item There has been an explosion in the amount of spatial data in recent years...
  \end{itemize} \vspace{3cm}
  \Put(60,0)  {\includegraphics[width=3cm, height=3cm]{Figures/autonomous_cars.jpg}}
\end{frame}

\begin{frame}[noframenumbering]{Introduction}
  \begin{itemize}
    \item There has been an explosion in the amount of spatial data in recent years...
  \end{itemize} \vspace{3cm}
  \Put(60,0)  {\includegraphics[width=3cm, height=3cm]{Figures/autonomous_cars.jpg}}
  \Put(100,40)  {\includegraphics[width=3cm, height=3cm]{Figures/survillance.png}}
\end{frame}

\begin{frame}[noframenumbering]{Introduction}
  \begin{itemize}
    \item There has been an explosion in the amount of spatial data in recent years...
  \end{itemize} \vspace{3cm}
  \Put(60,0)  {\includegraphics[width=3cm, height=3cm]{Figures/autonomous_cars.jpg}}
  \Put(100,40)  {\includegraphics[width=3cm, height=3cm]{Figures/survillance.png}}
  \Put(150,90){\includegraphics[width=3cm, height=3cm]{Figures/urban_planning.png}}
\end{frame}

\begin{frame}[noframenumbering]{Introduction}
  \begin{itemize}
    \item There has been an explosion in the amount of spatial data in recent years...
  \end{itemize} \vspace{3cm}
  \Put(60,0)  {\includegraphics[width=3cm, height=3cm]{Figures/autonomous_cars.jpg}}
  \Put(100,40)  {\includegraphics[width=3cm, height=3cm]{Figures/survillance.png}}
  \Put(150,90){\includegraphics[width=3cm, height=3cm]{Figures/urban_planning.png}}
  \Put(200,140){\includegraphics[width=3cm, height=3cm]{Figures/agricultural.png}}
\end{frame}

\begin{frame}{Introduction}
  \begin{itemize}
    \item The applications and commercial interest is clear...
  \end{itemize}
  \centering
  \includegraphics[width=0.75\linewidth]{Figures/apps.png}
\end{frame}

\begin{frame}{Introduction}
  \begin{itemize}
    \item But remember that ``Spatial is Special''...
  \end{itemize}
  \centering
  \includegraphics[clip, trim=0cm 5.75cm 0cm 0cm, width=0.75\linewidth]{Figures/logos.png}
\end{frame}

\begin{frame}[noframenumbering]{Introduction}
  \begin{itemize}
    \item But remember that ``Spatial is Special''...
  \end{itemize}
  \centering
  \includegraphics[clip, trim=0cm 2.4cm 0cm 0cm, width=0.75\linewidth]{Figures/logos.png}
\end{frame}

\begin{frame}[noframenumbering]{Introduction}
  \begin{itemize}
    \item But remember that ``Spatial is Special''...
  \end{itemize}
  \centering
  \includegraphics[clip, trim=0cm 0cm 0cm 0cm, width=0.75\linewidth]{Figures/logos.png}
\end{frame}

\begin{frame}{Introduction}
  \begin{itemize}
    \item Why do we need a new tool???
  \end{itemize}
  \centering
  \includegraphics[width=0.3\linewidth]{Figures/simba-logo.png}
\end{frame}

\begin{frame}{Introduction}
\begin{itemize}
\item Problems of Existing Systems...
  \begin{itemize}
    \item Single node database (low scalability) \\
	  \scriptsize{ArcGIS, PostGIS, Oracle Spatial.}
    \item \normalsize Disk-oriented cluster computation (low performance) \\
	  \scriptsize{Hadoop-GIS, SpatialHadoop, GeoMesa.}
    \item \normalsize No native support for spatial operators \\
	  \scriptsize{Spark SQL, MemSQL}
    \item \normalsize No sophisticated query planner and optimizer \\
	  \scriptsize{SpatialSpark, GeoSpark}
  \end{itemize}
\end{itemize}
\end{frame}

\begin{frame}{Introduction}
  \begin{itemize}
    \item Simba: \textbf{S}patial \textbf{I}n \textbf{M}emory \textbf{B}ig data \textbf{A}nalytics.
    \begin{enumerate}
     \item Extends Spark SQL to support spatial queries and offers simple APIs for both SQL and DataFrame.\pause
     \item Support two-layer spatial indexing over RDDs (low latency).\pause
     \item Designs a SQL context to run important spatial operations in parallel (high throughput).\pause
     \item Introduces spatial-aware and cost-based optimizations to select good spatial plans.
    \end{enumerate}
  \end{itemize}
\end{frame}

\begin{frame}{Introduction}
  \centering
  \includegraphics[clip, trim=1cm 20cm 10cm 1cm, page=3, width=0.9\linewidth]{simba_paper}
\end{frame}

\section{Simba Architecture Overview}

\begin{frame}[fragile]{Spark SQL Overview}
  Spark SQL is Apache Spark's module for working with structured data.\pause
  \begin{itemize}
   \item Seamlessly mixes SQL queries with Spark programs.\pause
   \item Connects to any data source the same way.\pause
   \item Includes a highly extensible cost-based optimizer (\textit{Catalyst}).\pause
   \item Spark SQL is a full-fledged query engine based on the underlying Spark core.
  \end{itemize}
\end{frame}

\begin{frame}[fragile]{Spark SQL Overview}
  \centering
  \begin{minted}[fontsize=\tiny,tabsize=8,breaklines,framesep=10pt,frame=single]{python}
	# Apply functions to results of SQL queries.
	context = HiveContext(sc)
	results = context.sql("""
				SELECT 
					* 
				FROM 
					people""")
	names = results.map(lambda p: p.name)
	# Query and join different data sources.
	context.jsonFile("s3n://...").registerTempTable("json")
	results = context.sql("""
				SELECT 
				      * 
				FROM 
				      people
				JOIN 
				      json ...""")
  \end{minted}
\end{frame}

\begin{frame}{Simba Architecture}
  Simba is an extension of Spark SQL across the system stack.
  \centering
  \begin{minipage}{.75\linewidth} 
    \includegraphics[clip, trim=2cm 21.75cm 11.1cm 1cm, page=4, width=\linewidth]{simba_paper}
  \end{minipage}
  \begin{minipage}{.21\linewidth} \tiny \color{white} Indexing \\ Optimizations \\ Spatial operations \\ Programming interface \end{minipage}  
\end{frame}

\begin{frame}[noframenumbering]{Simba Architecture}
  Simba is an extension of Spark SQL across the system stack.
  \centering
  \begin{minipage}{.75\linewidth} 
    \begin{tikzpicture}
      \node[anchor=south west,inner sep=0] at (0,0) {\includegraphics[clip, trim=2cm 21.75cm 11.1cm 1cm, page=4, width=\linewidth]{simba_paper}};
      \draw[blue,very thick,rounded corners] (4.4,1.85) rectangle (8.95,2.32);
      \draw[blue,very thick,rounded corners] (2.07,2.37) rectangle (4.04,2.83);
    \end{tikzpicture}
  \end{minipage}
  \begin{minipage}{.21\linewidth} \tiny \color{blue} Indexing \\ \color{white} Optimizations \\ Spatial operations \\ Programming interface \end{minipage}  
\end{frame}

\begin{frame}[noframenumbering]{Simba Architecture}
  Simba is an extension of Spark SQL across the system stack.
  \centering
  \begin{minipage}{.75\linewidth} 
    \begin{tikzpicture}
      \node[anchor=south west,inner sep=0] at (0,0) {\includegraphics[clip, trim=2cm 21.75cm 11.1cm 1cm, page=4, width=\linewidth]{simba_paper}};
      \draw[blue,very thick,rounded corners] (4.4,1.85) rectangle (8.95,2.32);
      \draw[blue,very thick,rounded corners] (2.07,2.37) rectangle (4.04,2.83);
      \draw[red,very thick,rounded corners] (4.08,2.37) rectangle (8.95,2.83);
    \end{tikzpicture}
  \end{minipage}
  \begin{minipage}{.21\linewidth} \tiny \color{blue} Indexing \\ \color{red} Optimizations \\ \color{white} Spatial operations \\ Programming interface \end{minipage}  
\end{frame}

\begin{frame}[noframenumbering]{Simba Architecture}
  Simba is an extension of Spark SQL across the system stack.
  \centering
  \begin{minipage}{.75\linewidth} 
    \begin{tikzpicture}
      \node[anchor=south west,inner sep=0] at (0,0) {\includegraphics[clip, trim=2cm 21.75cm 11.1cm 1cm, page=4, width=\linewidth]{simba_paper}};
      \draw[blue,very thick,rounded corners] (4.4,1.85) rectangle (8.95,2.32);
      \draw[blue,very thick,rounded corners] (2.07,2.37) rectangle (4.04,2.83);
      \draw[red,very thick,rounded corners] (4.08,2.37) rectangle (8.95,2.83);
      \draw[magenta,very thick,rounded corners] (0.03,2.88) rectangle (8.95,3.35);
    \end{tikzpicture}
  \end{minipage}
  \begin{minipage}{.21\linewidth} \tiny \color{blue} Indexing \\ \color{red} Optimizations \\ \color{magenta} Spatial operations \\ \color{white} Programming interface \end{minipage}  
\end{frame}

\begin{frame}[noframenumbering]{Simba Architecture}
  Simba is an extension of Spark SQL across the system stack.
  \centering
  \begin{minipage}{.75\linewidth} 
    \begin{tikzpicture}
      \node[anchor=south west,inner sep=0] at (0,0) {\includegraphics[clip, trim=2cm 21.75cm 11.1cm 1cm, page=4, width=\linewidth]{simba_paper}};
      \draw[blue,very thick,rounded corners] (4.4,1.85) rectangle (8.95,2.32);
      \draw[blue,very thick,rounded corners] (2.07,2.37) rectangle (4.04,2.83);
      \draw[red,very thick,rounded corners] (4.08,2.37) rectangle (8.95,2.83);
      \draw[magenta,very thick,rounded corners] (0.03,2.88) rectangle (8.95,3.35);
      \draw[olive,very thick,rounded corners] (0.03,3.4) rectangle (8.95,3.85);
    \end{tikzpicture}
  \end{minipage}
  \begin{minipage}{.21\linewidth} \tiny \color{blue}Indexing \\ \color{red} Optimizations \\ \color{magenta} Spatial operations \\ \color{olive} Programming interface \end{minipage}  
\end{frame}

\begin{frame}{Query Load in Simba}
  \centering
  \includegraphics[clip, trim=10.5cm 22.2cm 1cm 1cm, page=4, width=\linewidth]{simba_paper}
\end{frame}

\subsection{Programming Interface}

\begin{frame}[fragile]{Programming Interface}
  Support rich query types natively in the kernel...\\ \\
  \centering
  \begin{minipage}{.45\linewidth} 
  \begin{minted}[fontsize=\scriptsize,tabsize=5,breaklines,framesep=3pt,frame=single]{sql}
SELECT 
	*
FROM 
	points
SORT BY 
	(x - 2) * (x - 2) + 
	(y - 3) * (y - 3)
LIMIT 
	5
  \end{minted}
  \end{minipage} $\Longrightarrow$
  \begin{minipage}{.45\linewidth}  
  \begin{minted}[fontsize=\scriptsize,tabsize=5,breaklines,framesep=3pt,frame=single]{sql}  
SELECT 
	*
FROM 
	points
WHERE 
	POINT(x, y) IN 
		KNN(POINT(2, 3), 5)
  \end{minted}
  \end{minipage}
\end{frame}

\begin{frame}[fragile]{Programming Interface}
  Achieve something that is impossible in Spark SQL...\\ 
  \centering
  \begin{minted}[fontsize=\small,tabsize=6,breaklines,framesep=3pt,frame=single]{sql}
SELECT 
	*
FROM 
	queries q 
KNN JOIN 
	pois p
ON 
	POINT(p.x, p.y) IN KNN(POINT(q.x, q.y), 3)
  \end{minted}
\end{frame}

\begin{frame}[fragile]{Programming Interface}
  Fully compatible with standard SQL operators...\\ 
  \centering
  \begin{minted}[fontsize=\scriptsize,tabsize=6,breaklines,framesep=3pt,frame=single]{sql}
SELECT 
	p.id, count(*) AS n
FROM 
	pois AS p
DISTANCE JOIN 
	restaurants AS r
ON 
	POINT(r.lat, r.lng) IN CIRCLERANGE(POINT(p.lat, p.lng), 200.0)
GROUP BY 
	p.id
ORDER BY 
	n
  \end{minted}
\end{frame}

\begin{frame}[fragile]{Programming Interface}
  Same level of flexibility for DataFrames...\\ 
  \centering
  \begin{minted}[fontsize=\scriptsize,tabsize=6,breaklines,framesep=3pt,frame=single]{python}
pois.distanceJoin(restaurants, Point(pois("lat"), pois("lng")), 
			Point(restaurants("lat"), restaurants("lng")), 3.0)
	.groupBy(pois("id"))
	.agg(count("*").as("n"))
	.sort("n").show()
  \end{minted}
\end{frame}

\subsection{Indexing}

\begin{frame}{Table Indexing}
\begin{itemize}
 \item In Spark SQL:
 \begin{itemize}
  \item \texttt{Record} $\rightarrow$ \texttt{Row}
  \item  \texttt{Table} $\rightarrow$ \texttt{RDD[Row]}
 \end{itemize} \item Spark SQL makes a full scan of RDDs.
 \begin{itemize}
  \item Inefficient for spatial queries!!!
 \end{itemize}
 \item Solution: native \textbf{two-level} indexing over RDDs
\end{itemize}
\end{frame}

\begin{frame}{Table Indexing}
\begin{itemize}
 \item \textbf{\texttt{IndexRDD}}
 \begin{itemize}
  \item Pack all \texttt{Row} objects within a RDD partition into an array (O(1) cost for access).
  \item \texttt{IPartition} data structure:
  \begin{itemize}
   \item \mintinline{scala}{case class IPartition[Type](Data: Array[Type], I: Index)}
   \item \texttt{Index} can be \texttt{HashMap}, \texttt{TreeMap} or \texttt{RTree}.
  \end{itemize}
  \item So, by using \texttt{Type=Row}:
  \begin{itemize}
   \item \mintinline{scala}{type IndexRDD[Row] = RDD[IPartition[Row]]}
  \end{itemize}


 \end{itemize}

\end{itemize}
\end{frame}


\begin{frame}[fragile]{Two-level indexing strategy}
  \centering 
  \includegraphics[page=6,clip,trim=2cm 21.5cm 11.5cm 2.2cm,height=0.55\textheight]{simba_paper}
  \vspace{0.5cm}
  \begin{minted}[fontsize=\small,tabsize=6,breaklines,framesep=3pt,frame=single,escapeinside=~~,mathescape=true]{sql}
    CREATE INDEX idx_name ON R(~$x_1, ... , x_m$~) USE idx_type
    DROP INDEX idx_name ON table_name
  \end{minted}
\end{frame}

\subsection{Spatial Operations}

\subsection{Optimization}

\section{Experiments}

\section{Conclusions}

\begin{frame}{Conclusions}
  \begin{itemize}
    \item  
  \end{itemize}
\end{frame}

\begin{frame}{Future ideas}
  \begin{itemize}
    \item 
  \end{itemize}
\end{frame}


\subsection*{Thanks...}

\begin{frame}{}
  \centering
  \huge Thank you!!! \\
  \vspace{2cm}
  \large Do you have any question?
\end{frame}

\end{document}