%% LyX 2.1.4 created this file.  For more info, see http://www.lyx.org/.
%% Do not edit unless you really know what you are doing.
\documentclass{beamer}
\usepackage{minted}
\usepackage{animate}
\usepackage{graphicx}
\def\Put(#1,#2)#3{\leavevmode\makebox(0,0){\put(#1,#2){#3}}}
\usepackage{color}
\usepackage{tikz}
\usepackage{amssymb}

\newcommand\blfootnote[1]{%
  \begingroup
  \renewcommand\thefootnote{}\footnote{#1}%
  \addtocounter{footnote}{-1}%
  \endgroup
}

\definecolor{LightGray}{gray}{0.9}

\ifx\hypersetup\undefined
  \AtBeginDocument{%
    \hypersetup{unicode=true,
 bookmarksnumbered=false,bookmarksopen=false,
 breaklinks=false,pdfborder={0 0 0},colorlinks=false}
  }
\else
  \hypersetup{unicode=true,
 bookmarksnumbered=false,bookmarksopen=false,
 breaklinks=false,pdfborder={0 0 0},colorlinks=false}
\fi

\makeatletter
%%%%%%%%%%%%%%%%%%%%%%%%%%%%%% Textclass specific LaTeX commands.
 % this default might be overridden by plain title style
 \newcommand\makebeamertitle{\frame{\maketitle}}%
 % (ERT) argument for the TOC
 \AtBeginDocument{%
   \let\origtableofcontents=\tableofcontents
   \def\tableofcontents{\@ifnextchar[{\origtableofcontents}{\gobbletableofcontents}}
   \def\gobbletableofcontents#1{\origtableofcontents}
 }

%%%%%%%%%%%%%%%%%%%%%%%%%%%%%% User specified LaTeX commands.
\usetheme{Warsaw}
% or ...
\useoutertheme{infolines}
\addtobeamertemplate{headline}{}{\vskip2pt}

\setbeamercovered{transparent}
% or whatever (possibly just delete it)



\setbeamertemplate{footline}
{
  \leavevmode%
  \hbox{%
  \begin{beamercolorbox}[wd=.5\paperwidth,ht=2.25ex,dp=1ex,center]{title in head/foot}%
    \usebeamerfont{title in head/foot}\insertshorttitle
  \end{beamercolorbox}%
  \begin{beamercolorbox}[wd=.5\paperwidth,ht=2.25ex,dp=1ex,right]{date in head/foot}%
    \usebeamerfont{date in head/foot}\insertshortdate{}\hspace*{2em}
    \insertframenumber{} / \inserttotalframenumber\hspace*{2ex} 
  \end{beamercolorbox}}%
  \vskip0pt%
}

\makeatother

\begin{document}

\title[ ]{Towards Parallel Detection of Moving Flock Patterns in Large Spatiotemporal Datasets}
\author{Andres Calderon}
%\institute{University of California, Riverside}

\makebeamertitle

% \AtBeginSection[]{
%   \frame<beamer>{ 
%     \frametitle{Agenda}   
%     \tableofcontents[currentsubsection] 
%   }
% }

\newif\iflattersubsect

\AtBeginSection[] {
    \begin{frame}<beamer>
    \frametitle{Outline} %
    \tableofcontents[currentsection]  
    \end{frame}
    \lattersubsectfalse
}

\AtBeginSubsection[] {
    % \iflattersubsect
    \begin{frame}<beamer>
    \frametitle{Outline} %
    \tableofcontents[currentsubsection]  
    \end{frame}
    % \fi
    % \lattersubsecttrue
}

\section*{Introduction}

\begin{frame}{Trajectory Datasets}
  \begin{itemize}
    \item Sensors, sensors everywhere...
    \begin{itemize}
      \item Smart phones, GPS, RFID, WiFi, Bluetooth, IoT, Remote sensing...
    \end{itemize}
  \end{itemize}
  \centering
  \includegraphics[width=0.7\linewidth]{Figures/hurricanes.png}\blfootnote{\tiny \url{http://tinyurl.com/h4mbvxz}}
\end{frame}

\begin{frame}{Applications}
  \centering
  \includegraphics[width=\linewidth]{Figures/kaggle.png}\blfootnote{\tiny \url{http://tinyurl.com/jfm8qfu}}
\end{frame}

\begin{frame}{Applications}
  \centering
  \includegraphics[width=\linewidth]{Figures/microsoft.png}\blfootnote{\tiny \url{http://tinyurl.com/hpd4nxl}}
\end{frame}

\begin{frame}{Applications}
  \centering
  \includegraphics[width=\linewidth]{Figures/ebird.png}\blfootnote{\tiny \url{http://tinyurl.com/hkc6ahl}}
\end{frame}

\section{Moving Flock Patterns}

\begin{frame}{What is a flock???}
  \begin{definition}[$(\mu,\epsilon,\delta)-flock$]
    Sets of at least \alert{ $\mu$ } objects moving close enough (\alert{ $\epsilon$ }) for at least \alert{ $\delta$ } time intervals (Benkert et al, 2008). 
  \end{definition}
  \centering \includegraphics[height=0.6\textheight]{Figures/flock2.jpg}
\end{frame}

\begin{frame}{BFE algorithm \tiny (Vieira et al, 2009)}
  \centering \includegraphics[page=2,height=0.7\textheight]{Figures/flock/f0}
\end{frame}
\begin{frame}[noframenumbering]{BFE algorithm \tiny (Vieira et al, 2009)}
  \centering \includegraphics[page=2,height=0.7\textheight]{Figures/flock/f1}
\end{frame}
\begin{frame}[noframenumbering]{BFE algorithm \tiny (Vieira et al, 2009)}
  \centering \includegraphics[page=2,height=0.7\textheight]{Figures/flock/f3}
\end{frame}
\begin{frame}[noframenumbering]{BFE algorithm \tiny (Vieira et al, 2009)}
  \centering \includegraphics[page=2,height=0.7\textheight]{Figures/flock/f4}
\end{frame}
\begin{frame}[noframenumbering]{BFE algorithm \tiny (Vieira et al, 2009)}
  \centering \includegraphics[page=2,height=0.7\textheight]{Figures/flock/f5}
\end{frame}
\begin{frame}[noframenumbering]{BFE algorithm \tiny (Vieira et al, 2009)}
  \centering \includegraphics[page=2,height=0.7\textheight]{Figures/flock/f7}
\end{frame}
\begin{frame}[noframenumbering]{BFE algorithm \tiny (Vieira et al, 2009)}
  \centering \includegraphics[page=2,height=0.7\textheight]{Figures/flock/f8}
\end{frame}
\begin{frame}[noframenumbering]{BFE algorithm \tiny (Vieira et al, 2009)}
  \centering \includegraphics[page=2,height=0.7\textheight]{Figures/flock/f9}
\end{frame}
\begin{frame}[noframenumbering]{BFE algorithm \tiny (Vieira et al, 2009)}
  \centering \includegraphics[page=2,height=0.7\textheight]{Figures/flock/f10}
\end{frame}

\begin{frame}{Why am I doing this???}
    \begin{itemize}
     \item Why are moving flock patterns important?
     \begin{itemize}
      \item They capture the collective behavior of trajectories as groups.
     \end{itemize}
     \item Why is the finding of disks important?
     \begin{itemize}
      \item It is the base of the algorithm but it has a high complexity ($\mathcal{O}(2n^2)$).
      \item It is no trivial, disks can be at any location.
     \end{itemize}
    \end{itemize}
\end{frame}

\section{Implementation}
\begin{frame}{Demo}
    \begin{itemize}
     \item \large Demo time:
     \begin{itemize}
      \item \url{http://tinyurl.com/jl55849}.
     \end{itemize}
    \end{itemize}
\end{frame}

\section{Experiments}

\begin{frame}{Dataset}
  \begin{itemize}
    \item Beijing from Geolife project.
    \begin{itemize}
     \item 17 million points (no duplicates).
     \item 182 users in a period of over three years (from April 2007 to August 2012).
    \end{itemize}
    \item GDEL (Global Data on Events, Language and Tone)
    \begin{itemize}
     \item 75 Million records
     \item Seven attributes: timestamp, three two-dimensional coordinates (start, end and action of the event).
    \end{itemize}
    \item RC (Synthetic dataset)
    \begin{itemize}
     \item 1 Million to 1 Billion records, 2 to 6 dimensions.
     \item Clusters randomly generated using Gaussian distributions.
    \end{itemize}
  \end{itemize}
\end{frame}

\begin{frame}{Setup}
  \begin{itemize}
    \item 10 nodes cluster
    \item Processors: 6-core Intel Xeon E5 (1.6 to 2.0 GHz)
    \item RAM: 20 to 56 GB.
    \item Ubuntu 14.04 LTS, Hadoop 2.4.1, Spark 1.3.0
  \end{itemize}
\end{frame}

\begin{frame}{Datasets}
  \begin{itemize}
    \item OSM (OpenStreetMap)
    \begin{itemize}
     \item 2.2 Billion records, 132GB.
     \item Five fields: ID, a two-dimensional coordinate and two text information.
    \end{itemize}
    \item GDEL (Global Data on Events, Language and Tone)
    \begin{itemize}
     \item 75 Million records
     \item Seven attributes: timestamp, three two-dimensional coordinates (start, end and action of the event).
    \end{itemize}
    \item RC (Synthetic dataset)
    \begin{itemize}
     \item 1 Million to 1 Billion records, 2 to 6 dimensions.
     \item Clusters randomly generated using Gaussian distributions.
    \end{itemize}
  \end{itemize}
\end{frame}

\subsection{Comparison with Existing Systems}
\begin{frame}{Range and kNN Operations (OSM)}
  \centering
  %\includegraphics[page=10,clip,trim=11cm 23.1cm 2.5cm 1.8cm,width=\textwidth]{simba_paper}
\end{frame}


\subsection{Comparison against Spark SQL}

\begin{frame}{Range Query Performance (GDELT)}
  \centering
  %\includegraphics[page=12,clip,trim=2cm 19.3cm 11.3cm 5.3cm,width=\textwidth]{simba_paper}
\end{frame}


\subsection{Join Methods vs Dimensionality}

\begin{frame}{Join Operations Performance (RC)}
  \centering
  %\includegraphics[page=12,clip,trim=11.2cm 19.2cm 2.1cm 5.8cm,width=\textwidth]{simba_paper}
\end{frame}

\section{Conclusions}

\begin{frame}{Conclusions}
  \centering
  \Large Cooming soon...
\end{frame}

\subsection*{Thanks...}

\begin{frame}{}
  \centering
  \huge Thank you!!! \\
  \vspace{2cm}
  \large Do you have any question?
\end{frame}

\end{document}