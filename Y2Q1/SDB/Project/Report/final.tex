\documentclass[12pt]{scrartcl}
\usepackage[utf8]{inputenc}
\usepackage[square]{natbib}
\usepackage[hidelinks]{hyperref}
\usepackage{graphicx}
\usepackage{minted}

%opening
\title{Towards Parallel Detection of Moving Flock Patterns in Large Spatio-temporal Datasets\\ \normalsize{Final report}}
\author{Andres Calderon}

\begin{document}

\maketitle
 
\section{Introduction}
% In this section I plan to discuss background information and motivation about the topic.  Importance of trajectory datasets and complex motion patterns will be discussed.  I would like to introduce the notion of moving flock patterns and the challenges to find the set of disks together with some basic definitions.
% 
% The introduction gives a nice motivation and background for the proposed work. Try to make it more application-centric; that is, mention a few real-life applications that make use of the problem being addressed.  The introduction should also give the high-level idea of the solution and how it overcomes the limitations of existing work.  You might want to provide a final paragraph in the introduction that gives the overall report outline.

Spatio-temporal data is ubiquitous nowadays. Thanks to new technologies and location devices (such as Internet of Things, Remote Sensing, Smart phones, GPS, RFID, etc.), the collection of huge amount of spatio-temporal data is now possible. With the appearance of this datasets also appears the need of new techniques which allow the analysis and detection of useful patterns.  

Applications for this kind of information are diverse and interesting, in particular if they come in the way of trajectory datasets \citep{jeung_trajectory_2011, huang_mining_2015}. Case of studies range from transportation system management \citep{di_lorenzo_allaboard:_2016,johansson_efficiency_2015} to Ecology \citep{johnston_abundance_2015, la_sorte_convergence_2016}.  For instance, \cite{turdukulov_visual_2014} explore the finding of complex motion patterns to discover similarities between tropical cyclone paths.  Similarly, \cite{amor_persistence_2016} use eye trajectories to understand which strategies people use during a visual search. Also, \cite{holland_movements_1999} track the behavior of tiger sharks in the coasts of Hawaii in order to understand their migration patters.

Recently, there has been an increasing interest in exploiting more complex movement patterns in spatio-temporal datasets.  Traditional range and nearest neighbor queries do not capture the collective behavior of moving objects.  Moving cluster (\cite{kalnis_discovering_2005}), convoys (\cite{jeung_discovery_2008}) and flock patterns (\cite{benkert_reporting_2008, gudmundsson_computing_2006}) are new movement patterns which unveil how entities move together during a minimum time interval.  

In particular, a moving flock pattern show how objects move close enough during a given period of time.  A better understanding on how entities move in space is of special interest in areas such as sports (\cite{iwase_tracking_2002}),  surveillance and security (\cite{makris_path_2002,piciarelli_trajectory_2005}), urban development (\cite{huang_trajgraph:_2016, long_combining_2015}) and socio-economic geography (\cite{frank_life_2000}).

Despite the fact that much more data become available, state-of-the-art techniques to mine complex movement patterns still depict low scalability and poor performance in big spatial data.  The present work aims to find an initial solution to implement a parallel method to discover moving flock patterns in large spatio-temporal datasets.  It is thought that new trends in distributed in-memory framework for spatial operations could help to speed up the detection of this kind of patterns.

The following section will state the related work in the area.  Section \ref{sec:flock} will explain the details of the implementation of the proposed solution while section \ref{sec:experiments} will present their experimental results. Finally, section \ref{sec:conclusions} will discuss some conclusions and future work. 

\section{Related work}
% Some work about spatio-temporal patterns will be revisited in order to introduce some seminal work about moving flock patterns.  BFE algorithm must be presented in this section.  Similarly, I would like to discuss some advances in spatial data analysis tools in distributed platforms (in particular Simba).
% 
% The main goal of the related work section is to position your work in the literature of related work. If all the existing work in this problem is on a single machine, you can clearly say that your work is the first work in a distributed environment. If there are other work that run in a distributed environment, you will have to mention how your problem or solution is different.

Recently increase use of location-aware devices (such as GPS, Smart-phones and RFID tags) has allowed the collection of a vast amount of data with a spatial component linked to them.  Different studies have focused in analyzing and mining this kind of collections \citep{leung_knowledge_2010, miller_geographic_2001}.  In this area, trajectory datasets have emerged as an interesting field where diverse kind of patterns can be identified \citep{zheng_computing_2011, vieira_spatio-temporal_2013}.  For instance, authors have proposed techniques to discover motion spatial patterns such as moving clusters \citep{kalnis_discovering_2005}, convoys \citep{jeung_discovery_2008} and flocks \citep{benkert_reporting_2008, gudmundsson_computing_2006}.  In particular, \cite{vieira_-line_2009} proposed BFE (Basic Flock Evaluation), a novel algorithm to find moving flock patterns in polynomial time over large spatio-temporal datasets.  
 
A flock pattern is defined as a group of entities which move together for a defined lapse of time \citep{benkert_reporting_2008}.  Applications to this kind of patterns are rich and diverse.  For example, \citep{calderon_romero_mining_2011} finds moving flock patterns in iceberg trajectories to understand their movement behavior and how they related to changes in ocean's currents. 
 
The algorithm proposed by \citep{vieira_-line_2009} presents an initial strategy in order to detect flock patterns.  In that, first they find disks with a predefined diameter where moving entities could be close enough at a given time interval.  This is a costly operation due to the large number of points and intervals to be analyzed ($\mathcal{O}(2n^2)$ per time interval).  The technique uses a grid-based index and a stencil (see figure~\ref{fig:grid}) to speed up the process, but the complexity is still high.

\cite{calderon_romero_mining_2011} and \cite{turdukulov_visual_2014} use a frequent pattern mining approach to improve performance during the combination of disks between time intervals.  Similarly, \cite{tanaka_improved_2016} introduce the use of plane sweeping along with binary signatures and inverted indexes to speedup the same process.  However, the above-mentioned methods still keep the same strategy as BFE to find the disks at each interval.  

\citep{arimura_finding_2014} and \citep{geng_enumeration_2014} use depth-first algorithms to analyze the time intervals of each trajectory to report maximal duration flocks.  However, these techniques are not suitable to find patterns in an on-line fashion.

Given the high complexity of the task, it should not be surprising the use of parallelism to increase performance.  \cite{fort_parallel_2014} use extremal and intersection sets to report maximal, longest and largest flocks on the GPU with the limitations of its memory model.  

Indeed, despite the popularity of cluster computing frameworks (in particular whose supporting spatial capabilities \citep{eldawy_spatialhadoop:_2014, yu_demonstration_2016, pellechia_geomesa:_2015-1, xie_simba:_2016-1}) there are not significant advances in this area.  At the best of our knowledge, this work is the first to explore in-memory distributed systems towards the detection of moving flock patterns.

\begin{figure}[t]
 \centering
 \includegraphics[width=0.3\textwidth]{./figures/grid.png}
 \caption{Grid-based index used in \cite{vieira_-line_2009}.}
 \label{fig:grid}
\end{figure}

\section{Parallelizing flock detection}\label{sec:flock}
% I plan to explain the details of the algorithm implemented in Simba and how some spatial predicates introduced by it can leverage the finding of disks.
% 
% "Finding disks" seems to be very low-level for someone skimming over the report. You might want to give it a simpler title such as "Parallelizing Flock Detection" or something along these lines. It will be also a good idea to give a brief background on Simba in a separate section in case readers are not familiar with it. You only need to mention the parts that are relevant to your work.

Given that the finding of disks at each time interval is one of the most costly operations towards the detection of moving flock patterns, the main goal of this work is to implement a parallel method to detect that set of disks.  In order to do that, we will use the spatial operations offered by Simba \citep{xie_simba:_2016-1}, a distributed in-memory spatial analytics engine based on Apache Spark. This section explains the details of the algorithm implemented in Simba and how some spatial predicates introduced by it can leverage the finding of disks.

\subsection{Spatial operations on Simba}
Simba (Spatial In-Memory Big data Analytics) extends the Spark SQL engine to provide rich spatial operations through both SQL and the DataFrame API.  Besides, it introduces two-layer spatial indexing and cost-based optimizations to support efficient spatial queries.  Simba is open source and public available at the project homepage\footnote{\url{http://www.cs.utah.edu/~dongx/simba/}}.

In particular, \texttt{DISTANCE JOIN} and \texttt{CIRCLERANGE} operators were used in order to find groups of points lying close enough each other.  The algorithm uses a user-defined distance ($\varepsilon$) to define the diameter of the disk.  Figure \ref{fig:sql} shows the SQL statement used to find pairs of points inside an $\varepsilon$.

\begin{figure}[t]
 \centering
    \begin{minted}[fontsize=\footnotesize,tabsize=8,breaklines,framesep=10pt,frame=single, escapeinside=||,mathescape=true]{sql}
SELECT 
	* 
FROM 
	points p1
|\color{blue}{DISTANCE JOIN}|
	points p2 
ON 
	POINT(p2.x, p2.y) IN |\color{blue}{CIRCLERANGE}|(POINT(p1.x, p1.y), |$\varepsilon$|)
WHERE 
	p1.id < p2.id
    \end{minted}
 \caption{SQL statement on Simba to find points lying inside an $\varepsilon$ distance.}
 \label{fig:sql}
\end{figure}




\section{Experiments}\label{sec:experiments}
% The setup and details about the datasets I plan to use will be discussed in this section.  It will show some figures comparing the implementation with a sequential version of the algorithm.
% 
% This part was very generic in your report outline. You need to be more specific on which datasets you will use and the types of experiments you want to run. Mainly, I'm interested to know the questions you aim to answer in the experiments section.


\section{Conclusions and future work}\label{sec:conclusions}
% I will share the lessons learned during the project and some future ideas to continue with the research.


\bibliographystyle{plainnat}
\bibliography{flock}

\end{document}
