\documentclass[10pt]{scrartcl}
\usepackage[utf8]{inputenc}
\usepackage{geometry}
\usepackage{booktabs}
\usepackage{graphicx}
\usepackage{hyperref}
\hypersetup{
colorlinks=false,
hidelinks
}
\usepackage{minted}
\usepackage{xcolor}
\usepackage{lineno}
\usepackage{tikz}
\usetikzlibrary{tikzmark}


\begin{document}

\section*{Improving disk filtering by maximal pattern detection.}
After the initial phase of BFE algorithm, it is assumed that a valid set of candidates disk has been generated.  It is a simple list with a generated key for each disk and a list with the ID's of the moving objects which the disk encloses. The coordinates of the disk are not needed for filtering purposes but they should be useful if further visualizations are planned.

After the candidate disks detection, additional filtering should be carried.  BFE algorithm states two filtering process: Firstly, remove any disk which do not group a minimum number of moving object (defined by $\mu$ parameter). Secondly, remove any disk whose element set is a subset of another disk's element set. It means that If from two disks one of them contains all the other's elements, just the former should be kept.  If two disks share the same elements, just one of them should remain.  



\end{document}
